\documentclass[runningheads]{llncs}

\usepackage[T1]{fontenc}

\usepackage{cite}
\usepackage{amsmath,amssymb,amsfonts}
\usepackage{algorithmic}
\usepackage{textcomp}
\usepackage{graphicx}
\usepackage{xcolor}
\usepackage{hyperref}
\usepackage{booktabs}
\usepackage{float}
\usepackage[utf8]{inputenc}
\usepackage[spanish]{babel}

\def\BibTeX{{\rm B\kern-.05em{\sc i\kern-.025em b}\kern-.08em
    T\kern-.1667em\lower.7ex\hbox{E}\kern-.125emX}}

\begin{document}
%
\title{Sistema Multi-Robot para Planificación de Cobertura de Área y Patrullaje de Perímetro}
%
\author{Daniel García Burgos,
Hongfan Yang,
José Antonio García Campanario,
José Manuel Parejo Alonso,
José Francisco López Ruiz}
%
\authorrunning{D.G. Burgos, H. Yang, J.A. García, J.M. Parejo, J.F. López}
%
\institute{Máster en Ingeniería Electrónica, Robótica y Automática \\
Universidad de Sevilla, Sevilla, España}
%
\maketitle
%
\begin{abstract}
En este artículo se presenta el desarrollo e implementación de un sistema multi-robot para tareas de planificación de cobertura (Coverage Path Planning) y patrullaje de perímetro. El sistema utiliza el algoritmo DARP (Divide Areas Algorithm for Robotics Platforms) para la división eficiente del área de trabajo entre múltiples agentes, asegurando una cobertura completa y optimizada. Además, se propone una estrategia para la vigilancia de perímetros. La implementación se ha realizado sobre ROS2 (Robot Operating System 2), integrando el piloto automático PX4 y el simulador Gazebo para la validación de los algoritmos en entornos realistas. Se detallan la arquitectura del sistema, la metodología empleada y se presentan resultados de simulación que demuestran la viabilidad y eficacia de la solución propuesta.

\keywords{Planificación de cobertura \and Sistemas multi-robot \and ROS2 \and PX4 \and DARP \and Patrullaje de perímetro \and Simulación.}
\end{abstract}

\section{Introducción}
\label{sec:introduccion}
% - Contexto de la robótica móvil y sistemas multi-robot
% - Importancia del Coverage Path Planning (CPP) y patrullaje
% - Objetivos del proyecto
% - Estructura del paper

\section{Estado del Arte}
\label{sec:estado_arte}
% - Algoritmos relevantes
% - Enfoques para multi-robot
% - Caso de cobertura de área
% - Caso de cobertura de perímetro

\section{Algoritmos de Planificación}
\label{sec:algoritmos}
\subsection{Cobertura de Área con DARP}
% - Descripción del algoritmo DARP
% - Adaptaciones realizadas

\subsection{Patrullaje de Perímetro}
% - Descripción del algoritmo de patrullaje
% - Lógica de funcionamiento

\section{Descripción del Sistema}
\label{sec:sistema}
\subsection{Arquitectura de Software}
% - Uso de ROS2 Humble
% - Nodos implementados y comunicación
% - Estructura del repositorio (mencionar brevemente Docker y herramientas)

\subsection{Plataforma de Simulación}
% - PX4 Autopilot
% - Gazebo Garden
% - Modelos de robots utilizados

\subsection{Interfaz Gráfica (HUD)}
% - Características

\section{Simulación y Resultados}
\label{sec:resultados}
\subsection{Entorno de Simulación}
% - Descripción de los parámetros
% - Configuración de los experimentos

\subsection{Resultados Experimentales}
% - Métricas de evaluación (tiempo de cobertura, etc.)
% - Gráficas y tablas de resultados
% - Imágenes de la simulación (capturas de Gazebo/RViz)

\subsection{Discusión}
% - Análisis de los resultados obtenidos
% - Comparativa entre diferentes configuraciones o escenarios

\section{Conclusiones y Trabajo Futuro}
\label{sec:conclusiones}
% - Resumen de los logros alcanzados
% - Limitaciones detectadas
% - Líneas de trabajo futuro (mejoras en algoritmos, pruebas en hardware real, etc.)

%
% ---- Bibliography ----
%
\bibliographystyle{splncs04}
\bibliography{bibliography}

\end{document}